\section{Eigenwerte}
Im gesamten Kapitel sei $X$ ein linearer Raum über dem Körper $\mathbb{K}$.
\subsection{Determinanten}
Wir beginnen mit einem Exkurs über Permutationen.  Dazu sei $(S_n,\circ )$ die in \hyperref[symmetrische]{Beispiel \ref*{symmetrische}} eingeführte symmetrische Gruppe aller bijektiven Selbstabbildungen von $\{1,\cdots ,n\},\ n\in\mathbb{N}$.  Ihre Elemente $\sigma$ werden als \underline{Permutationen} bezeichnet (von $\{1,\cdots ,n\}$) und man notiert sie als Schema:
\[\begin{bmatrix}1 & 2 & \cdots & n\\ \sigma (1) & \sigma (2) & \cdots & \sigma (n)\end{bmatrix}\]
oder als $n-$Tupel ($\sigma (1),\cdots ,\sigma (n)$).  Weil $\sigma$ bijektiv ist, kommt jede Zahl $j\in\{1,\cdots ,n\}$ genau einmal in $(\sigma (1),\cdots ,\sigma (n))$ vor; ferner gibt es genau $n!$ solche Permutationen.
\subsubsection{Definition (Signum)}
Es sei $\sigma \in S_n$ und $s(\sigma )$ bezeichnet die Anzahle der Paare $(i,j)\in\mathbb{N}^2$ mit $1\leq i<j\leq n$ und $\sigma (i) > \sigma (j)$.  Dann ist das \underline{Signum} einer Permutation $\sigma$ definiert durch
\[\mathrm{sgn~}\sigma := (-1)^{s(\sigma )}\]
\subsubsection{Beispiel}
\label{4.1.2}
Für die identische Permutation id$=(1,2,\cdots ,n)$ gilt $s(\sigma )=0$ und folglich sgn~id$=1$.  Weiter erhält man
\begin{align*}
\sigma &=(2,1,3,4,\cdots ,n),\ s(\sigma )=1,\ \mathrm{sgn~}\sigma =-1\\
\sigma &=(n,n-1,\cdots ,2, 1),\ s(\sigma )=\frac{(n-1)n}{2},\mathrm{sgn~}\sigma =(-1)^{s(\sigma )}
\end{align*}
\subsubsection{Proposition}
Für alle $\sigma ,\tau \in S_n$ gilt sgn~$\sigma \circ \tau =$ sgn~$\sigma\cdot$sgn~$\tau$.
\subsubsection{Bemerkung}
Mittels \hyperref[4.1.2]{Beispiel \ref*{4.1.2}} ist $1=$sgn~id$=$sgn~$\sigma \circ \sigma ^{-1}$ und damit erhalten wir
\[\label{4.1a} (4.1.a)\ \mathrm{sgn~}\sigma ^{-1}=\mathrm{sgn~}\sigma\text{ für alle }\sigma \in S_n\]
\underline{Beweis}: Es seien $\sigma ,\tau \in S_n$ und $x_1,\cdots ,x_n\in\mathbb{Q}$ paarweise verschieden.  Dann sind auch $y_i:=x_{\sigma (i)}$ mit $1\leq i\leq n$ paarweise verschieden.
\renewcommand{\labelenumi}{(\Roman{enumi})}
\begin{enumerate}
\item Zunächst gilt die Identität
\[\label{4.1b} (4.1b)\ \mathrm{sgn~}\sigma =\prod _{1 \leq i<j \leq n} \frac{x_{\sigma (i)}-x_{\sigma (j)}}{x_i -x_j}\]
denn Zähler und Nenner des Produkts stimmen bis auf ihr Vorzeichen überein.  Im Zähler tritt ein Faktor $x_k -x_l$ mit $k>l$ genau $s(\sigma )$-mal auf, während dies im Nenner nicht vorkommt.
\item Aufgrund von \hyperref[4.1b]{$4.1b$} erhalten wir
\[\mathrm{sgn~}\tau =\prod _{1\leq i<j\leq n} \frac{y_{\tau (i)} -y_{\tau (j)}}{y_i - y_j}=\prod _{1\leq i<j\leq n} \frac{x_{\sigma \circ \tau (i)}-x_{\sigma \circ \tau (j)}}{x_{\sigma (i)}-x_{\sigma (j)}}\]
\end{enumerate}
und folglich resultiert die Behauptung aus
\begin{align*}
\mathrm{sgn~} \sigma\circ\tau &\stackrel{(4.1b)}{=} \prod _{1\leq i<j\leq n}\frac{x_{\sigma \circ \tau (i)}-x_{\sigma\circ\tau (j)}}{x_i -x_j}\\
&= \left(\prod _{1\leq i<j\leq n} \frac{x_{\sigma\circ\tau (i)}-x_{\sigma\circ\tau (j)}}{x_{\sigma (i)}-x_{\sigma (j)}}\right) \left(\prod _{1\leq i<j\leq n}\frac{x_{\sigma (i)}-x_{\sigma (j)}}{x_i-x_j}\right)\\
&= \mathrm{sgn~}\tau \cdot \mathrm{sgn~}\sigma
\end{align*}
\subsubsection{Definition (Determinante)}
Die durch det:$\mathbb{K}^{n\times n}\rightarrow \mathbb{K}$
\[\phantomsection\label{4.1c}(4.1c)\ \mathrm{det}A:=\sum _{\sigma\in S_n} \mathrm{sgn~}\sigma \prod _{i=1}^na_{i\sigma (i)}\]
\subsubsection{Bemerkung}
\renewcommand{\labelenumi}{(\arabic{enumi})}
\begin{enumerate}
\item \hyperref[4.1c]{$(4.1c)$} heißt auch \underline{Leibniz-Formel}
\item Es gilt die Beziehung $\mathrm{det}(\alpha A)=\alpha ^n\cdot\mathrm{det}A$ für alle $\alpha\in\mathbb{K},A\in\mathbb{K}^{n\times n}$, folglich ist die Determinante für $n\geq 2$ nicht linear.
\end{enumerate}
\subsubsection{Beispiel}
Wir erhalten det$0=0$ und det$I_n=1$
\subsubsection{Beispiel}
In Dimensionen $n\leq 3$ kann die Determinante einer Matrix $A\in\mathbb{K}^{n\times n}$ verhältnismäßig einfach berechnet werden.
\begin{enumerate}
\item Für $n=1$ gilt det$A=a_{1,1}$
\item Für $n=2$ ist $S_2=\{(1,2),(2,1)\}$ und wir erhalten det$A=$det$\begin{pmatrix}a_{1,1} & a_{1,2}\\ a_{2,1} & a_{2,2}\end{pmatrix}=a_{1,1}a_{2,2}-a_{2,1}a_{1,2}$
\item Für $n=3$ gilt $S_3=\{(1,2,3),(2,3,1),(3,1,2),(2,1,3),(3,2,1),(1,3,2)\}$ wobei die ersten drei Permutationen das Signum $1$ besitzen und die weiteren das Signum $-1$ besitzen.  Dies liefert die  \underline{Regel von Sarrus}
\begin{align*}
\mathrm{det}A&=\mathrm{det}\begin{pmatrix}a_{1,1} & a_{1,2} & a_{1,3}\\ a_{2,1} & a_{2,2} & a_{2,3}\\ a_{3,1} & a_{3,2} & a_{3,3}\end{pmatrix} \left|\begin{matrix}a_{1,1} & a_{1,2}\\ a_{2,1} & a_{2,2}\\ a_{3,1} & a_{3,2}\end{matrix} \leftarrow \text{Bildlich dargestellt wie die Regel funktioniert}\right|\\
&=a_{1,1}a_{2,2}a_{3,3}+a_{1,2}a_{2,3}a_{3,1}+a_{1,3}a_{2,1}a_{3,2}-a_{1,1}a_{2,3}a_{3,2}-a_{1,2}a_{2,1}a_{3,3}-a_{1,3}a_{2,2}a_{3,1}
\end{align*}
\end{enumerate}
\subsubsection{Lemma}
Für alle $A,B\in\mathbb{K}^{n\times n}$ gilt
\renewcommand{\labelenumi}{(\alph{enumi})}
\begin{enumerate}
\item det$A=$det$A^T$
\item Entsteht $B$ durch eine Permutation $\sigma \in S_n$ der Spalten von $A$ (d.h. ist formal $B=(a_{\sigma (1)}\,a_{\sigma (2)},\cdots ,a_{\sigma (n)})$), oder der Zeilen von $A$ $\left(\text{d.h. formal }B=\begin{pmatrix}a^{\sigma (1)}\\ \vdots \\ a^{\sigma (n)}\end{pmatrix}\right)$,\\
so gilt det$B=$sgn~$\sigma$det$A$
\item Falls zwei Spalten der Zeilen von $A$ übereinstimmen, so ist det$A=0$.
\end{enumerate}
\underline{Beweis}: Es seien $A,B\in\mathbb{K}^{n\times n}$
\begin{enumerate}
\item Durch direktes Nachrechnen und \hyperref[4.1a]{$4.1a$} folgt 
\begin{align*}
\mathrm{det}A^T&\stackrel{(4,1c)}{=}\sum _{\sigma\in S_n}\mathrm{sgn~}\sigma \prod _{i=1}^n a_{\sigma (i)i}\\
&=\sum _{\sigma \in S_n} \mathrm{sgn~}\sigma ^{-1}\prod _{j=1}^n a_{j\sigma ^{-1}(j)}\\
&= \mathrm{det}A
\end{align*}
\item folgt ähnlich und (c) ist etwas involvierter.
\end{enumerate}
Eine zentrale Eigenschaft von Determinanten ist ihre Multiplikativität.  Allerdings ist sie in Dimensionen $n>1$ nicht additiv: Beispiel det$(I_n+I_n)=2^n$, aber det$I_n=1$.
\subsubsection{Satz (Multiplikativität der Determinante)}
\label{4.1.10}
Es gilt \[\phantomsection\label{4.1d}(4.1d)\mathrm{det}(AB)=\mathrm{det}A\cdot\mathrm{det}B\text{ für alle }A,B\in\mathbb{K}^{n\times n}.\]
Zusätzlich zu Satz 3.3.14: Charakterisierung regulärer Matrizen: 
\subsubsection{Satz (Regularität und die Determinante}
\label{4.1.11}
Eine Matrix $A\in\mathbb{K}^{n\times n}$ ist genau dann regulär, wenn det$A\not= 0$.  Dann gilt
\[\phantomsection\label{4.1e}(4.1e)\ \mathrm{det}A^{-1}=\frac{1}{\mathrm{det}A}\]
\subsubsection{Korollar}
\label{4.1.12}
Für ähnliche Matrizen $A,B\in\mathbb{K}^{n\times n}$ ist det$A=$det$B$.
\subsubsection{Bemerkung}
Es sei $X$ ein linearer Raum, mit dim$X<\infty$.  Auf Basis von \hyperref[4.1.12]{Korollar \ref{4.1.12}} lässt sich auch die \underline{Determinante} det$:L(X)\rightarrow \mathbb{K}$ einer linearen Abbildung $T\in L(X)$.  Mit ihrer darstellenden Matrix $T_\mathcal{X}$ ist
\[\mathrm{det}T:=\mathrm{det}T_\mathcal{X}\]
Hierbei ist det$T$ unabhängig von der Basis $\mathcal{X}$, denn nach \hyperref[3.3.16]{Satz \ref{3.3.16}} sind alle darstellenden Matrizen ähnlich und haben gleiche Determinante.\\
\underline{Beweis}:  Mit $A,B\in\mathbb{K}^{n\times n}$ und einer regulären Matrix $S\in\mathbb{K}^{n\times n}$ mit $B=S^{-1}AS$ gilt aufgrund von \hyperref[4.1.10]{Satz \ref{4.1.10}} und \hyperref[4.1.11]{Satz \ref{4.1.11}}
\[\mathrm{det}B=\mathrm{det}S^{-1}AS\stackrel{\hyperref[4.1d]{(4.1d)}}{=}\mathrm{det}S^{-1}\mathrm{det}A\mathrm{det}S=\mathrm{det}A\]
Problem mit der Leibniz \hyperref[4.1c]{(4.1c)}: Sehr aufwändig!\\
Lösung: Zu gegebenem $A\in\mathbb{K}^{n\times n}$ und $1\leq k,l\leq n$ sei $A_{kl}:=(a_{ij})_{\substack{1\leq i\leq n, i\not=k\\ 1\leq j\leq n, i\not=l}}\in\mathbb{K}^{(n-1)\times n(-1)}$ diejenige Matrix, welche aus $A$ durch Streichen der $k-$ten Zeile und der $l-ten$ Spalte entsteht.
\subsubsection{Proposition (Entwicklung von det)}
\label{4.1.14}
Es sei $A\in\mathbb{K}^{n\times n}$.  Für alle Indizen $k,l\in\{1,\cdots ,n\}$ gilt dann:
\renewcommand{\labelenumi}{(\alph{enumi})}
\begin{enumerate}
\item die \underline{Entwicklung nach der $k-$ten Zeile}
\[\mathrm{det}A=\sum _{i=1}^n (-1)^{k+j}a_{kj}\mathrm{det}A_{kj}\]
\item die \underline{Entwicklung nach der $l-$ten Spalte}
\[\mathrm{det}A=\sum _{i=1}^n (-1)^{i+l}a_{il}\mathrm{det}A_{il}\]
\end{enumerate}
\subsubsection{Beispiel}
Durch Entwicklung nach der ersten Zeile erhalten wir:
\begin{align*}
\mathrm{det}\begin{pmatrix}0 & 1 & 2\\ 3 & 4 & 5\\ 6 & 7 & 8\end{pmatrix}&=0\mathrm{det}\begin{pmatrix}4 & 5\\ 7 & 8\end{pmatrix}+(-1)\mathrm{det}\begin{pmatrix}3 & 5\\ 6 & 8\end{pmatrix}+2\mathrm{det}\begin{pmatrix}3 & 4\\ 6 & 7\end{pmatrix}\\
&=0
\end{align*}
Entwicklung nach der ersten Spalte liefert entsprechend:
\begin{align*}
\mathrm{det}\begin{pmatrix}0 & 1 & 2\\ 3 & 4 & 5\\ 6 & 7 & 8\end{pmatrix}&=0\mathrm{det}\begin{pmatrix}4 & 5\\ 7 & 8\end{pmatrix}+(-1)3\mathrm{det}\begin{pmatrix}1 & 2\\ 7 & 8\end{pmatrix}+6\mathrm{det}\begin{pmatrix}1 & 2\\ 4 & 5\end{pmatrix}\\
&=0
\end{align*}
\subsubsection{Beispiel (Dreiecksmatrizen)}
Ist $A\in\mathbb{K}^{n\times n}$ eine Dreiecksmatrix, so gilt det$A=\prod _{i=1}^n a_{ii}$.
\subsubsection{Proposition (Inverse und det)}
Ist $A\in\mathbb{K}^{n\times n}$ regulär.
\[A^{-1}=\frac{1}{\mathrm{det}A}\left( (-1)^{i+j}\mathrm{det}A_{ij}\right)_{1\leq j,i\leq n}\]
\subsubsection{Beispiel}
In $\mathbb{K}^{2\times 2}$ gilt
\[\begin{pmatrix}a_{11} & a_{12}\\ a_{21} & a_{22}\end{pmatrix}^{-1}=\frac{1}{a_{11}a_{22}-a_{21}a_{12}}\begin{pmatrix}a_{22} & -a_{12}\\ -a_{21} & a_{11}\end{pmatrix}\]
\underline{Beweis}: Mittels der Matrix $B:=((-1)^{i+j}\mathrm{det}A_{ji})_{1\leq i,j\leq n}$ erhalten wir durch Nachrechnen $AB=BA=\mathrm{det}A\cdot I_n$.  Wegen \hyperref[3.3.12]{Korollar \ref{3.3.12}} ist (det$A)^{-1}B$ die Inverse von $A$.
\subsection{Eigenwerte und Eigenvektoren}
Es sei $X$ ein linearer Raum über $\mathbb{K}$.\\
\underline{Ziel}: Finde zu $T\in L(X)$ eine Basis $\mathcal{X}$ von $X$ derart, dass $T_\mathcal{X}\in\mathbb{K}^{n\times n}$ möglichst "`einfach"' ist.  Hilfreich sind hierbei diejenigen Vektoren, welche von $T$ auf ein Vielfaches abgebildet werden.
\subsubsection{Definition (Eigenwert, Eigenvektor und Eigenraum)}
\label{4.2.1}
Existiert zu $T\in L(X)$ ein Skalar $\lambda\in\mathbb{K}$ und ein Vektor $x\in X\setminus \{0 \}$ mit
\[\phantomsection\label{4.2a}(4.2a)\ Tx=\lambda x\]
So nennt man $x$ den zum \underline{Eigenwert} $\lambda$ gehörigen \underline{Eigenvektor} von $T$.  Der Kern $E_\lambda := N(T-\lambda \mathrm{id}x)$ wird \underline{Eigenraum} von $T$ und dessen Dimension die \underline{geometrische Vielfachheit} von $\lambda$ genannt.  Das \underline{Spektrum} $\sigma (T)\subseteq \mathbb{K}$ ist die Menge aller Eigenwerte.
\subsubsection{Bemerkung}
\label{4.2.2}
\renewcommand{\labelenumi}{(\arabic{enumi})}
\begin{enumerate}
\item Eine Abbildung $T\in L(X)$ besitzt genau dann einen nichttrivialen Kern, falls $0\in \sigma (T)$ gilt.  Im Fall dim$X < \infty$ ist $T$ genau dann invertierbar, wenn $0\not\in\sigma (T)$.
\item Für jedes $\lambda \in \sigma (T)$ ist der zugehörige Eigenraum $E_\lambda$ \underline{invariant} bezüglich $T$, d.h.
\[x\in E_\lambda \Rightarrow Tx\in E_\lambda\]
Ist insbesondere $m:=$dim$E_\lambda<\infty$, so besitzt $S:=T\mid _{E_\lambda} \in L(E_\lambda)$ bezüglich jeder Basis $\mathcal{X}$ von $E_\lambda$ die Darstellung $S_\mathcal{X}=\mathrm{diag}(\lambda ,\cdots ,\lambda )\in\mathbb{K}^{m\times m}$.
\item Mit $A\in\mathbb{K}^{n\times n}$ besteht das Spektrum $\sigma (T_A)$ aus allen $\lambda\in\mathbb{K}$ derart, dass der Lösungsraum der homogenen Gleichung $[A-\lambda I_n]x=0$ nichttrivial ist; letzterer stimmt mit $E_\lambda$ überein.
\end{enumerate}
\subsubsection{Beispiel}
\begin{enumerate}
\item Mit der Nullabbildung $0\in L(X)$ gilt $\stackrel{\in L(X)}{0}x=\stackrel{\in X}{0}$ für alle $x\in X$.  Folglich ist $\sigma (0) = \{0\}$, jedes $x\not=0$ ist Eigenvektor und $E_0=X$.  Für die Identität id$x$ gilt $\sigma (\mathrm{id}x)=\{1\}$ und $E_1=X$.
\item Wir betrachten die von $A=\begin{pmatrix}1 & 2\\ 2 & 1\end{pmatrix}$ induzierte Abbildung $T_A\in L(\mathbb{R}^2)$:
\begin{itemize}
\item Wegen $T_A\begin{pmatrix}-1\\ 1\end{pmatrix}=-1\begin{pmatrix}-1\\ 1\end{pmatrix}$ ist $-1$ ein Eigenwert, $\begin{pmatrix}-1\\ 1\end{pmatrix}$ zugehörigen Eigenvektor und $E_{-1}=\mathbb{R}\begin{pmatrix}-1\\ 1\end{pmatrix}$ Eigenraum zu $-1$; also hat $-1$ die geometrische Vielfachheit $1$.
\item Aufgrund von $T_A\begin{pmatrix}1\\ 1\end{pmatrix}=3\begin{pmatrix}1\\ 1\end{pmatrix}$ ist ferner $3$ ein Eigenwert $\begin{pmatrix}1\\ 1\end{pmatrix}$ zugehöriger Eigenvektor und $E_3=\mathbb{R}\begin{pmatrix}1\\ 1\end{pmatrix}$.  Geometrische Vielfachheit ist $1$.
\end{itemize}
\end{enumerate}
\subsubsection{Beispiel (Shift-Operator)}
Auf dem Folgenraum $X:=F(\mathbb{Z},\mathbb{K})$ betrachten wir den Vorwärts-Shift $(Tx)_k :=x_{k+1},\ T\in L(X)$.  Im Fall $\lambda \not= 0$ gilt die Eigenwert - Eigenvektor - Beziehung \hyperref[4.2a]{$(4.2a)$} genau dann, wenn
\[[x_{k+1}=(Tx)_k=\lambda x_k],\text{ für alle } k\in\mathbb{Z}\]
dies ist wiederum für jede Folge $x^\lambda \in X,\ x_k^\lambda =\lambda ^k$ erfüllt.  Im Fall $\lambda = 0$ gibt es dagegen keine Folge $X\not= 0$, welche \hyperref[4.2a]{$(4.2a)$} erfüllt.  Daher besitzt der Shift-Operator $T$ das Spektrum $\sigma (T)=\mathbb{K}\setminus \{0\}$ und $x^\lambda$ ist ein zu $\lambda \in \sigma (T)$ gehöriger Eigenvektor.  Aufgrund von $E_\lambda =span\{x^\lambda\}$ besitzt jedes $\lambda$ die geometrische Vielfachheit $1$.
\subsubsection{Proposition}
\label{4.2.5}
Sind $\lambda _i,\ 1\leq i\leq m$, paarweise verschiedene Eigenwerte von $T\in L(X)$ mit zugehörigen Eigenvektoren $x_i\in X$, so ist $\{x_1,\cdots ,x_m\}$ linear unabhängig.\\
\underline{Beweis}: Induktion über $m$: Im Fall $m=1$ ist wegen $x_1\not=0$ nichts zu zeigen.  Es gelte nun die Aussage für $m$ und wir machen den Ansatz 
\[\phantomsection\label{4.2b}(4.2b)\sum _{i=1}^{m+1}\xi _i x_i=0\text{ mit } \xi _1,\cdots ,\xi _{m+1}\in\mathbb{K}\]
Es resultiert hieraus die Beziehungen
\[0\stackrel{\hyperref[4.2b]{(4.2b)}}{=}T(\sum _{i=1}^m \xi _i x_i)\stackrel{\hyperref[3.1a]{(3.1a)}}{=}\sum _{i=1}^{m+1}\xi _i Tx_i\stackrel{\hyperref[4.2a]{(4.2a)}}{=}\sum _{i=1}^{m+1}\xi _i\lambda _i x_i\]
\[0=\lambda _{m+1}\sum _{i=1}^{m+1}\xi _i x_i = \sum _{i=1}^{m+1} \xi _i \lambda _{m+1} x_i\]
und durch Subtraktion folgt $0=\sum _{i=1}^m \xi _i (\lambda _i-\lambda _{m+1}) x_i$.  Laut Induktionsannahme ist $\{x_1,\cdots ,x_m\}$ linear unabhängig und damit $\xi _i (\lambda _i -\lambda _{m+1})=0$.  Da die Eigenwerte paarweise verschieden sind, folgt zunächst $\xi _i=0,\ 1\leq i\leq m$.  Mit \hyperref[4.2b]{$(4.2b)$} folgt $\xi _{m+1} x_{m+1}=0$.  Also Eigenvektor ist $x_{m+1}\not=0$ und daher $\xi _{m+1}=0$.
\subsubsection{Korollar}
Ist $n=$dim$X<\infty$, so gilt:
\renewcommand{\labelenumi}{(\alph{enumi})}
\begin{enumerate}
\item $T$ hat höchstens $n$ verschiedene Eigenwerte.
\item Besitzt $T$ genau $n$ verschiedene Eigenwerte $\lambda _i$ mit zugehörigen Eigenvektoren $x_i$, so ist $\mathcal{X}:=\{x_1,\cdots ,x_n\}$ eine Basis von $X$ und $T_\mathcal{X}=diag(\lambda _1,\cdots ,\lambda _n)$.
\end{enumerate}
\underline{Beweis}:
\begin{enumerate}
\item Hätte $T$ mehr als $n$ verschiedene Eigenwerte, so gäbe es mehr als $n$ linear unabhängige Vektoren in $X$.
\item Ergibt sich aus \hyperref[2.4.15]{Korollar \ref{2.4.15}} und \hyperref[4.2.2]{Bemerkung \ref{4.2.2}(2)}
\end{enumerate}
\subsection{Das charakteristische Polynom}
\textbf{Im Folgendem gilt: $\chi =$griechischer Buchstabe "`chi"', und $\mathcal{X}$ ein Skript-$X$\\}
$X$ sei linearer Raum über $\mathbb{K}$, dim$X=n<\infty$, $\mathcal{X}$ sei Basis.  Dann hat jede $T\in L(X)$ eine darstellende Matrix $T_\mathcal{X}\in \mathbb{K}^{n\times n}$.  Bezeichnet $\mathcal{X}'$ eine weitere Basis von $X$, so folgern wir aus \hyperref[3.3.16]{Satz \ref{3.3.16}} die Existenz einer regulären Basiswechselmatrix $S$ mit $T_{\mathcal{X}'}=S^{-1}T_\mathcal{X}S$ und nach den Sätzen \hyperref[4.1.10]{\ref{4.1.10}}, \hyperref[4.1.11]{\ref{4.1.11}}:
\[\mathrm{det}(T_{\mathcal{X}'}-tI_n)=\mathrm{det}(S^{-1}T_\mathcal{X}S-tS^{-1}S)\stackrel{\hyperref[4.1d]{(4.1d)}}{=}\mathrm{det}S^{-1}\mathrm{det}(T_\mathcal{X}-tI_n)\mathrm{det}S\stackrel{\hyperref[4.1e]{(4.1e)}}{=}\mathrm{det}(T_\mathcal{X}-tI_n)\]
für alle $t\in \mathbb{K}$.  Also hängen die Werte von $t\mapsto \mathrm{det}(T_\mathcal{X}-tI_n)$ nur von $T\in L(X)$ und nicht von $\mathcal{X}$ ab.
\subsubsection{Definition (charakteristisches Polynom)}
Besitzt $T\in L(X)$ eine darstellende Matrix $T_\mathcal{X}\in\mathbb{K}^{n\times n}$, so ist das \underline{charakteristisches Polynom} von T gegeben durch $\chi _T(t)=$det$(T_\mathcal{X}-tI_n)$.
\subsubsection{Bemerkung}
Es seien $A\in\mathbb{K}^{n\times n}$ und die induzierte Abbildung $T_A\in L(\mathbb{K}^n)$ gegeben.
\renewcommand{\labelenumi}{(\arabic{enumi})}
\begin{enumerate}
\item Im Fall $X=\mathbb{K}^n$ bezeichnet man $\chi _{T_A}$ auch als charakteristisches Polynom von $A$.
\item Die \underline{Spur} von $A$ ist definiert durch (vgl. Aufgabe)
\[trA=\sum _{i=1}^n a_{ii}\]
und mit $\chi _{T_A}(t)=c_nt^n+\cdots +c_1t+c_0$ gilt $c_n=(-1)^n$, $c_{n-1}=(-1)^{n-1}trA$, $c_0=$det$A$.
\end{enumerate}
Man sagt ein Polynom $\chi \in P_n(\mathbb{K})$ \underline{zerfällt in Linearfaktoren} falls es Koeffizienten $\mu \in \mathbb{K},\ \lambda _1,\cdots ,\lambda _n\in\mathbb{K}$ gibt, mit
\[\phantomsection\label{4.3a}(4.3a)\chi (t) =\mu \prod _{i=1}^n (t-\lambda _i)\text{ auf } \mathbb{K}\]
Die Werte $\lambda _i$ sind dann die Nullstellen von $\chi$, ihre \underline{algebraische Vielfachheit} gibt an, wie oft der \underline{Linearfaktor} $t-\lambda _i$ in \hyperref[4.3a]{$(4.3a)$} vorkommt.
\subsubsection{Satz}
Für jedes $T\in L(X)$ gilt $\sigma (T)=\chi _T^{-1}(\{0\})$, und die Vielfachheit von $\lambda \in \sigma (T)$ heißt \underline{algebraische Vielfachheit} des Eigenwerts $\lambda$.\\
\underline{Beweis}: Mittels \hyperref[4.2.1]{Definition \ref{4.2.1}} gelten die Äquivalenzen
\[\lambda \not\in \sigma (T)\Leftrightarrow E_\lambda = N(T-\lambda \mathrm{id}\ x)=\{0\}\Leftrightarrow[T_\mathcal{X}-\lambda I_n]x=0\text{ hat nur die triviale Lösung.}\]
\[\Leftrightarrow T_\mathcal{X}-\lambda I_n\text{ ist regulär (\hyperref[3.3.14]{Satz \ref{3.3.14}})}\]
\[\Leftrightarrow \chi _T(\lambda)=\mathrm{det}(T_\mathcal{X}-\lambda I_n)\not= 0\text{ nach \hyperref[4.1.11]{Satz \ref{4.1.11}}}\]
\subsubsection{Beispiel}
\label{4.3.4}
Es seien $\alpha \in [0,2\pi ),\ A=\begin{pmatrix}cos \alpha & -sin \alpha\\ sin\alpha & cos\alpha \end{pmatrix}$
\begin{enumerate}
\item Über dem Körper $\mathbb{K}=\mathbb{R}$ betrachten wir $T=T_A\in L(\mathbb{R}^2)$.  Sie hat das charakteristische Polynom $\chi _T(t)=t^2-2cos\alpha \cdot t+1$.  Für $\alpha =0$ ist $\sigma (T)=\{1\}$ und für $\alpha = \pi$ gilt $\sigma (T)=\{-1\}$.  Die Eigenwerte sind doppelte Nullstellen von $\chi _T$ und damit von algebraischer Vielfachheit $2$.  Ansonsten besitzt $\chi _T$ für $\alpha \not\in \{0,\pi\}$ keine reellen Nullstellen, d.h. $\sigma (T)=\emptyset$.
\item Mit $\mathbb{K}=\mathbb{C}$ besitzt $T=T_A\in L(\mathbb{C}^2)$ für alle $\alpha$ das Spektrum
\[\sigma (T)=\{cos\alpha -i\sqrt{1-cos^2\alpha }, cos\alpha +i\sqrt{1-cos^2\alpha }\}.\]
Seine Elemente haben für $\alpha\not\in \{0,\pi \}$ die algebraische Vielfachheit $1$.
\end{enumerate}
Wir nennen einen Körper $\mathbb{K}$ \underline{algebraisch abgeschlossen}, falls jedes Polynom $p\in P(\mathbb{K})$ mit deg$p>0$ mindestens eine Nullstelle hat.
\subsubsection{Proposition}
Ist $\mathbb{K}$ algebraisch abgeschlossen, so hat jedes $T\in L(X)$ einen Eigenwert.
\subsubsection{Beispiel}
\begin{enumerate}
\item $\mathbb{R}$ ist nicht algebraisch abgeschlossen (z.B. $t^2 +1$).  Der Fundamentalsatz der Algebra besagt gerade, dass $\mathbb{C}$ algebraisch abgeschlossen ist; folglich zerfällt auch jedes Polynom über $\mathbb{C}$ in Linearfaktoren.
\item $\mathbb{Q}$ ist nicht algebraisch abgeschlossen (z.B. $t^2-2$).
\end{enumerate}
\subsubsection{Satz}
\label{4.3.7}
Die geometrische Vielfachheit jedes Eigenwertes ist kleiner oder gleich seiner algebraischen Vielfachheit.\\
\underline{Beweis}: Es sei $T\in L(X)$ und $\lambda \in \sigma (T)$ mit $m:=$dim$E_\lambda$.  Dann existieren $m$ linear unabhängige Eigenvektoren von $T$ in $E_\lambda$.  Wählt man diese als ersten Teil einer Basis $\mathcal{X}$ von $X$, so gilt (vgl. \hyperref[4.2.2]{Bemerkung \ref{4.2.2}(2)})
\[T_\mathcal{X} =\begin{pmatrix}\Lambda & B\\ 0 & C\end{pmatrix}, \Lambda=diag(\lambda ,\cdots ,\lambda )\in\mathbb{K}^{n\times m},\ B\in\mathbb{K}^{m\times (m-n)},\ C\in\mathbb{K}^{(n-m)\times (n-m)}\]
und wir erhalten nach \hyperref[4.1.14]{Satz \ref{4.1.14}}, dass
\[\chi _T(t)=(\lambda -t)^m \mathrm{det}(C-tI_{n-m})\]
Daher ist die algebraische Vielfachheit von $\lambda$ mindestens $m$.
\subsection{Diagonalisierung und Trigonalisierung}
Wir betrachten $n-$dimensionale lineare Räume $X$ mit $n<\infty$.
\subsubsection{Definition (diagonalisierbar, trigonalisierbar)}
Eine Abbildung $T\in L(X)$ heißt
\renewcommand{\labelenumi}{(\alph{enumi})}
\begin{enumerate}
\item \underline{diagonalisierbar} falls eine Basis $\mathcal{X}$ von $X$ existiert, in der $T_\mathcal{X}$ eine Diagonalmatrix ist.
\item \underline{trigonalisierbar}, falls eine Basis $\mathcal{X}$ von $X$ existiert, in der $T_\mathcal{X}$ obere Dreiecksform besitzt.
\end{enumerate}
\subsubsection{Bemerkung}
\renewcommand{\labelenumi}{(\arabic{enumi})}
\begin{enumerate}
\item Entsprechend wird eine Matrix $A\in \mathbb{K}^{n\times n}$ \underline{diagonalisierbar (trigonalisierbar)} genannt, falls $T_A\in L(\mathbb{K}^n)$ diese entsprechende Eigenschaft besitzt, bzw. ähnlich zu einer Diagonalmatrix (obere Dreiecksmatrix) ist.
\item Mit oberen Dreiecksmatrizen zu arbeiten ist reine Konvention und keine mathematische Notwendigkeit.
\end{enumerate}
\subsubsection{Satz (Charakterisierung von Diagonalisierbarkeit)}
\label{4.4.3}
Eine Abbildung $T\in L(X)$ ist genau dann Diagonalisierbar, wenn gilt:
\renewcommand{\labelenumi}{(\roman{enumi})}
\begin{enumerate}
\item Ihr charakteristisches Polynom zerfällt in Linearfaktoren.
\item Die geometrische Vielfachheit jedes Eigenwertes stimmt mit seiner algebraische Vielfachheit überein.
\end{enumerate}
\underline{Beweis}:
Wir zeigen zwei Richtungen:
\renewcommand{\labelitemi}{}
\begin{itemize}
\item ($\Rightarrow$) Es sei $T\in L(X)$ diagonalisierbar und $\mathcal{X}$ eine Basis von $X$ derart, dass \[T_\mathcal{X}=diag(a_1,\cdots ,a_n)\text{ mit }a_1,\cdots ,a_n\in\mathbb{K}.\]  Das charakteristisches Polynom ergibt sich daher zu 
\[\phantomsection\label{4.4a}(4.4a)\ \chi _T=\prod _{k=1}^n (a_k -t)\left[=\mathrm{det}(T_\mathcal{X}-tI_n)=\mathrm{det}\begin{pmatrix}a_1-t & \ & \ \\ \ & \ddots & \ \\ \ & \ & a_n-t\end{pmatrix}\right]\]
und zerfällt offenbar in Linearfaktoren.  Nach \hyperref[4.3.7]{Satz \ref{4.3.7}} bleibt also nachzuweisen, dass die geometrische Vielfachheit eines $\lambda \in \sigma (T)$ mindestens gleich seiner algebraischen Vielfachheit ist.  Bezeichnet r die Algebraische Vielfachheit von $\lambda$, so kommt der Linearfaktoren $\lambda -t$ genau $r-$mal in \hyperref[4.4a]{$(4.4a)$} vor, d.h. $r$ Diagonalelemente von $T_\mathcal{X}$ sind gleich $\lambda$.  Damit werden $r$ Elemente der Basis $\mathcal{X}$ auf ihr $\lambda -$faches abgebildet und folglich ist dim$E_\lambda \geq r$
\item ($\Leftarrow$) Umgekehrt zerfalle $\chi _T$ in Linearfaktoren, es gelte die Bedingung (ii) und es sei $\sigma (T)=\{\lambda _1,\cdots,\lambda _m\}$ mit paarweise verschiedenen Eigenwerten $\lambda _i,\ 1\leq i\leq m\leq n$.  Zu jedem Eigenraum $E_{\lambda _i}$ wählen wir eine Basis
\[\mathcal{X}_i=\{x_1^i,\cdots ,x_{r_i}^i\}\text{ mit }1\leq i\leq m,r_1+\cdots r_m=n.\]
und erhalten somit $n$ Vektoren im Raum $X$.  Wir wollen zeigen, dass diese linear unabhängig sind.  Dazu 
\[\phantomsection\label{4.4b}(4.4b)\ 0=\sum _{i=1}^m \sum _{j=1}^{r_i}\xi _j^ix_j^i=\sum _{i=1}^m y_i\text{ mit } y_i=\sum _{j=1}^{r_i}\xi _j^i x_j^i\text{ und Koeffizienten }\xi _j^i\]
Die Invarianz der $E_{\lambda _i}$ zeigt, dass jedes solcher $y_i\in E_{\lambda _i}$ auf sein $\lambda _i$-faches abgebildet wird, also Eigenvektor von $T$ ist.  Mit \hyperref[4.2.5]{Proposition \ref{4.2.5}} liefert dies die lineare Unabhängigkeit von $\{y_1,\cdots ,y_m\}$.  Dann muss aber $0=y_i=\sum _{j=1}^{r_1}\xi _j^ix_j^i,\ 1\leq i\leq m$ gelten, denn andernfalls wäre \hyperref[4.4b]{$(4.4b)$} eine nichttriviale Darstellung der $0$.  Da $\mathcal{X}_i$ Basis ist, folgt $\xi _j^i=0$ und somit sind die $n$ Vektoren aus $\mathcal{X}_1,\cdots ,\mathcal{X}_n$ linear unabhängig, bilden also eine Basis von $X$, in welcher $T_\mathcal{X}$ diagonal ist.
\end{itemize}
\subsubsection{Beispiel}
\label{4.4.4}
\numbers
\begin{enumerate}
\item Wir betrachten die lineare Abbildung 
\[T\in L(\mathbb{R}^3),\ Tx=\begin{pmatrix}0 & 1 & -1\\ 3 & 2 & -3 \\ 2 & 2 & -3\end{pmatrix}x\]
und dem charakteristischen Polynom $\chi _T(t)=t^3+t^2-t-1=(t+1)^2(t-1)$.  Es zerfällt in Linearfaktoren und der Eigenwert $-1$ besitzt die algebraische Vielfachheit $2$.  Der Lösungsraumen der homogenen linearer Gleichung $[T+id]x=0$ lautet
\[E_{-1}=span\left\{\begin{pmatrix}-1\\ 1\\ 0\end{pmatrix},\begin{pmatrix}1\\ 0\\ 1\end{pmatrix}\right\}\]
und ist der zugehörige Eigenraum zum Eigenwert$-1$.  Somit hat $-1$ die geometrische Vielfachheit $2$.  Der weitere Eigenwert $1$ ist algebraisch einfach.  Seine geometrische Vielfachheit $1$ ergibt sich aus der Gleichung $[t-id]x=0$, deren Lösungsraum
\[E_1=span\left\{\begin{pmatrix}1\\ 3\\ 2\end{pmatrix}\right\}\]
gerade der zum Eigenwert $1$ gehörige Eigenraum ist.  Deshalb garantiert \hyperref[4.4.3]{Satz \ref{4.4.3}} die Diagonalisierbarkeit von $T$.  In der Tat, mit der aus den Eigenvektoren von $T$ gebildeten Basiswechselmatrix $S=\begin{pmatrix}-1 & 1 & 1\\ 1 & 0 & 3 \\ 0 & 1 & 2\end{pmatrix}$ (gebildet von $E_{-1}$ und $E_1$ als Spalten) gibt $S^{-1}TS=\begin{pmatrix}-1 & 0 & 0 \\0 & -1 & 0\\ 0 & 0 & 1\end{pmatrix}$.
\item Wir betrachten $T\in L(\mathbb{R}^3)$
\[Tx:=\begin{pmatrix}-2 & 1 & 3\\ 2 & 1 & -1\\ -7 & 2 & 7\end{pmatrix}x\]
mit dem charakteristischen Polynom $\chi _T(t)=-(t-2)^3$, das in Linearfaktoren zerfällt.  Der Eigenwert 2 von $T$ hat die algebraische Vielfachheit $3$.  Man verifiziert leicht $E_2=N(T-2id)=\mathbb{R}\begin{pmatrix}1\\ 1\\ 1\end{pmatrix}$,
weshalb die geometrische Vielfachheit $1$ ist.  Also ist $T$ nicht diagonalisierbar.
\end{enumerate}
\subsubsection{Satz (Charakterisierung von Trigonalisierbarkeit)}
\label{4.4.5}
Eine Abbildung $T\in L(X)$ ist genau dann trigonalisierbar, wenn ihr charakteristisches Polynom in Linearfaktoren zerfällt.
\subsubsection{Bemerkung (Jordan-Normalform)}
Der \hyperref[4.4.5]{Satz \ref{4.4.5}} lässt sich wesentlich präziser fassen - wenn auch mit deutlich aufwändigeren Beweis: zu jedem $T\in L(X)$, deren charakteristisches Polynom in Linearfaktoren zerfällt, existiert eine Basis $\mathcal{X}$ von $X$, so dass die darstellende Matrix $T_\mathcal{X}$ in \underline{Jordan-Normalform} ist.
\[T_\mathcal{X}=diag(J_1,\cdots ,J_r)\text{ mit } 1\leq i\leq r\]
Hierbei besitzt jeder \underline{Jordan-Block} die Form 
\[J_i=\begin{pmatrix}\lambda _i & 1\\ & \lambda _i & 1 \\& & \ddots & 1\\ & & & \lambda _i\end{pmatrix}\in\mathbb{K}^{n_i\times m_i},\ 1\leq i\leq r\]
mit $\sigma (T) = \{\lambda _1,\cdots ,\lambda _r\}$ und $n_1+\cdots +n_r=n$.
\subsubsection{Beispiel}
\numbers
\begin{enumerate}
\item Mit $\alpha \in [0,2\pi )$ und der aus \hyperref[4.3.4]{Beispiel \ref{4.3.4}} bekannten Matrix:
\[A:=\begin{pmatrix}cos \alpha & -sin\alpha\\ sin\alpha & cos\alpha\end{pmatrix}\in\mathbb{R}^{2\times 2}\]
betrachten wir die induzierte lineare Abbildung $T=T_A\in L(\mathbb{R}^2)$.  Ihr charakteristisches Polynom $\chi _T(t):=t^2-2cos(\alpha )t+1$ hat die Diskriminante $4(cos^2\alpha -1)$.  Daher ist $T$ genau für $\alpha \in \{0,\pi \}$ trigonalisierbar über $\mathbb{R}$.
\item Weiter betrachten wir die in \hyperref[4.4.4]{Beispiel \ref{4.4.4}(2)} diskutierte Abbildung $T\in L(\mathbb{R}^3)$
\[Tx:=\begin{pmatrix}-2 & 1 & 3\\ 2 & 1 & -1\\ -7 & 2 & 7\end{pmatrix}X\]
Da ihr charakteristisches Polynom $\chi _T(t)=(t-2)^3$ in Linearfaktoren zerfällt, ist sie nach \hyperref[4.4.5]{Satz \ref{4.4.5}} trigonalisierbar.  In der Tat liefert die Matrix $S:=\begin{pmatrix}-3 & -4 & 1\\ -3 & 2 & 0\\ -3 & -7 & 0\end{pmatrix}$ ihre Jordan-Normalform $J=S^{-1}TS=\begin{pmatrix}2 & 1 & 0\\ 0 & 2 & 1\\ 0 & 0 & 2\end{pmatrix}$
\end{enumerate}
\subsubsection{Korollar}
Jede lineare Abbildung $T\in L(X)$ auf einem endlich-dimensionalen Raum $X$ über $\mathbb{C}$ ist trigonalisierbar.\\
\underline{Beweis}: Mit dem Fundamentalsatz der Algebra zerfällt jedes Polynom $p\in P(\mathbb{C})$ in Linearfaktoren, also insbesondere $\chi _T$.
\subsubsection{Korollar}
Zerfällt $\chi _T$ in Linearfaktoren, so gilt
\[\mathrm{det}T=\prod _{j=1}^n \lambda _j,\ tr\ T=\sum _{j=1}^n \lambda _j\text{ mit }\sigma (T)=\{\lambda _1,\cdots ,\lambda _n\}.\]
\underline{Beweis}:  Nach \hyperref[4.4.5]{Satz \ref{4.4.5}} ist $T\in L(X)$ trigonalisierbar und folglich ist jede Darstellung $T_\mathcal{X}$ ähnlich zu einer Dreiecksmatrix $D\in \mathbb{K}^{n\times n}$ vermöge $S$.  Dann folgt mit \hyperref[4.1.12]{Korollar \ref{4.1.12}}, dass det$T=$det$D=\lambda _1,\cdots ,\lambda _n$.  Nach einer Übungsaufgabe ist $trT=trS^{-1}(DS)=tr(DS)S^{-1}=trD=\lambda _1+\cdots +\lambda _n$.\\
Wir definieren die \underline{Potenzen} $T^k\in L(X),\ k\in\mathbb{N}_0$, einer Abbildung $T\in L(X)$ rekursiv.
\[T^0:=id\ x,\ T^{k+1}:=T\circ T^k\text{ für }k\in\mathbb{N}_0\]
und entsprechend für Matrizen $A\in \mathbb{K}^{n\times n}$.  Wegen dim$L(X)=n^2$(\hyperref[3.3.5]{Korollar \ref{3.3.5}}) ist die $(n^2+1)$-elementige Menge $\{T^0, T^1,\cdots ,T^{n^2}\}$ linear abhängig.  Tatsächlich hat sogar $\{T^0,\cdots ,T^n\}$ diese Eigenschaft:
\subsubsection{Satz (von Cayley-Hamilton)}
Jede lineare Abbildung $T\in L(X)$ erfüllt ihr charakteristisches Polynom, d.h. formal $\chi _T(T)=0$.
\subsubsection{Bemerkung}
\begin{enumerate}
\item Bezeichnet $\mathcal{X}$ eine Basis von $X$, so gilt für die darstellende Matrix $\chi _T(T_\mathcal{X})=0$
\item Für jedes $A\in\mathbb{K}^{2\times 2}$ gilt $A^2=(trA)A-(\mathrm{det}A)I_2$
\end{enumerate}